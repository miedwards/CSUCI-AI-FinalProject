\documentclass{article}
\usepackage[usenames,dvipsnames]{pstricks}
\usepackage{amsfonts}
\usepackage{amsmath}
\usepackage{amssymb}
\usepackage{epsfig}
\usepackage{graphicx}
\usepackage{mathrsfs}
\usepackage{pst-grad} % For gradients
\usepackage{pst-plot} % For axes
\usepackage{subcaption}
\usepackage{tikz}
\usepackage{fancyhdr}
\usepackage{algorithm}
\usepackage{algorithmic}
\usepackage{url}

\pagestyle{fancy}
\lhead{COMP 569}
\chead{Mark I. Edwards}
\rhead{Final Project}

\title{COMP 569 Final Project\\Image Recognition on Busy Images with Convolutional Neural Networks}
\author{Mark I. Edwards}

\begin{document}
\maketitle
\abstract{Convolutional Neural Networks (CNNs) are one of the most up-and coming
image recognition algorithm. In this project, I aim to use CNNs to perform image
recognition on busy images, locating small targets in images containing many
background patterns. I do this by replicating the work in~\cite{saito_2016} and
use and use CNNs to perform image recognition on aerial images of vehicles.}

\section*{Introduction}
Image recognition on busy images can be thought of as the task of teaching
computers to play Where's Waldo.\textsuperscript{TM} It is the task of teaching computers to
identify faces within a crowd, to identify animals within a landscape, or to
identify cars, buildings and roads in an aerial image. In this project, the
author examines the task of identifying cars within aerial images, using the
Vehicle Detection in Aerial Imagery (VEDAI) dataset~\cite{vedai_2015}.

\section*{Background}
Prior to the rise of Convolutional Neural Networks, a number of other approaches
were used to perform image recognition on busy images. One such method was known
as template matching. Some templates extract features, such as edges, curves or
points, and others consist of small pixel areas that are compared to patches
within the image containing the target to be matched with some similarity
measure~\cite{perveen_2013}. Such rudimentary learning methods and other expert
systems were popular through the 1960s and 1070s~\cite{minh_thesis}.

Starting in the 1980s, simple machine learning classifiers gained
popularity~\cite{minh_thesis}. One of the most popular machine learning
algorithms used for analyzing aerial images was the Naive Bayes classifier.
Early approaches attempted to classify every pixel $i$ into a class $c+i$ in the
input image based on local features $x_i$. The primary drawback of this is that
it requires that we specify the class-conditional distribution $p(x_i \mid c_i =
k)$ for each class $k$. The multivariate normal distribution is typically used,
but this limits the classifier to only linear or quadratic decision boundaries.
Neural Networks rose as an alternative to Bayesian models as they can
approximate the class-conditional distribution as a differentiable function with
learned parameters, both eliminating the restriction on decision boundaries and
the requirement that the class-conditional distribution be explicitly found. 

Another method that gained traction was the use of machine learning
``ensembles.'' Machine learning ensembles combine multiple machine learning
models, typically either through Boosting or Averaging. Boosting is the process
of using multiple weak machine learning algorithms in succession, to gradually
refine results. A common boosting algorithm is the AdaBoost ``Adaptive
Boosting'' method, which integrates multiple Tree learning algorithms in
succession. AdaBoost is itself a stage within one of the best known image
recognition algorithms, the Viola-Jones algorithm. An averaging algorithm
instead uses multiple weak classifiers in parallel, averaging the output. One
example of a commonly used averaging algorithm is the Random Forest method. In
the Random Forest method, multiple decision trees are trained on randomly
selected subsets of the training data taken with replacement using a random
subset of features in a process known as ``Bagging.'' The predictions of the
different trees are then averaged to get the final result. The random forest
algorithm has shown considerable success at detecting brain tumors within
MRI scans~\cite{brats}.

To improve further upon the successes of ensemble methods the authors
of~\cite{saito_2016} have turned to Neural Networks. Deep Neural Networks
have a structure that effectively performs boosting and averaging on
neurons following the perceptron algorithm. In particular, Convolutional Neural
Networks (CNNs) have shown considerable prowess at image detection.

\section*{Convolutional Neural Networks}
Neural Networks are a paradigm inspired by the function of the biological brain.
Artificial Neurons simulate the function of biological neurons using a
weighted sum of their inputs and an activation function $f$ as shown
in Figure~\ref{fig:bioArtNeuron} and Figure~\ref{fig:activation}.
\begin{figure}[h]
\centering
\caption{Biological and Artificial Neurons}
\includegraphics[width=0.99\textwidth]{./biological_artificial_neurons.png}
\label{fig:bioArtNeuron}
\end{figure}

\begin{figure}[h]
\centering
\caption{Activation Functions}
\includegraphics[width=0.99\textwidth]{./activation.png}
\label{fig:activation}
\end{figure}

However, individual neurons are only capable of categorization along linear
boundaries. In order to achieve the non-linearity needed for more accurate
predictions, multiple neurons must be connected into a network as shown in
Figure~\ref{fig:neuralNetwork}. In fact, a sufficiently large Neural Network can
approximate any continuous function on compact subsets of
$\mathbb{R}^n$~\cite{hornik_1991}.

\begin{figure}[h]
\centering
\caption{Neural Network}
\includegraphics[width=0.5\textwidth]{./neural_network.png}
\label{fig:neuralNetwork}
\end{figure}

However, fully connected Deep Neural Networks are computationally inefficient
and prone to overfitting. This is because in a fully connected neural network,
each neuron in the first hidden layer (the layer after the input layer) receives
information about each pixel in the image, which is often computatationally
untenable and extraneous. Convolutional Neural Networks employ ``spatial
sparsity'' in the form of convolutional layers. 

\begin{figure}[h]
\centering
\caption{Convolution}
\includegraphics[width=0.5\textwidth]{./conv_diagram.png}
\label{fig:kernel}
\end{figure}

In the field of Image Processing, the word convolution or kernel refers to a linear
operation that takes place over a small region within a larger image as shown in
Figure~\ref{fig:kernel}. In a convolutional neural network, such convolutions
take the form of connections within the network corresponding to a region. In
essence, the convolutions are learned rather than pre-determined. 

Typically, CNNs employ multiple learned convolutions in their convolutional
layers. This results in much more data than the original input, which adds to
memory and processing costs. Since some of this data is extraneous, a form of
sub-sampling called ``pooling'' is often employed. One popular form of pooling is
``max pooling.'' In max pooling, many small spatial regions of the output of
multiple convolutions are selected. From these regions, only the largest value
is kept. This reduces computational cost and reduces overfitting, without much
loss of information.

\begin{figure}[h!]
\centering
\caption{Minh Net}
\includegraphics[width=0.99\textwidth]{./MinhNet.png}
\label{fig:minh}
\end{figure}

The exact CNN used in this project is shown in Figure~\ref{fig:minh}. It was
developed in~\cite{saito_2016} with inspiration from~\cite{minh_thesis}. It employs a
large initial convolution that allows it to closely match small targets. It was
originally applied to the Massachusetts Roads Dataset and the Massachusetts
Buildings Dataset. However, the author of this project was unable to access
those databases, and instead used the Vehicle Detection in Aerial Imagery (VEDAI)
dataset~\cite{vedai_2015}. In addition, many training methods implemented
within~\cite{saito_2016} were not implemented, such as model averaging over
spatial displacement, to account for small changes in target position, and data
augmentation by randomly rotating the input images. As implemented, this CNN
takes a $512 \times 512$ 3 color image as input, and outputs a $16\times 16$
grid with a pixel depth equal to one more than the number of classifications (an
additional background classification).


\newpage
%\medskip
%\begin{thebibliography}{9}
%\bibitem{lifeofsun}
% \bibitem{latexcompanion} 
% Michel Goossens, Frank Mittelbach, and Alexander Samarin. 
% \textit{The \LaTeX\ Companion}. 
% Addison-Wesley, Reading, Massachusetts, 1993.
%  
% \bibitem{einstein} 
% Albert Einstein. 
% \textit{Zur Elektrodynamik bewegter K{\"o}rper}. (German) 
% [\textit{On the electrodynamics of moving bodies}]. 
% Annalen der Physik, 322(10):891–921, 1905.
%  
% \bibitem{knuthwebsite} 
% Knuth: Computers and Typesetting,
% \\\texttt{http://www-cs-faculty.stanford.edu/\~{}uno/abcde.html}
%\end{thebibliography}
\bibliographystyle{acm}
\bibliography{final_project}


\end{document}
